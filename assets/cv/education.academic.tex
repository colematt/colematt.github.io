%!TEX encoding = UTF8
%!TEX root =cv.tex

\begin{rubric}{Education}
\entry*[May 2024 (expected)] {%
\textbf{Ph.D. Computer Science}, State University of New York at Binghamton University, Binghamton, New York.
\par Dissertation: \textit{Enforcing Integrity Models Through Hardware-Software Cohesive Systems}}
%
\entry*[August 2018] {%
\textbf{M.Sc. Computer Science}, State University of New York at Binghamton University, Binghamton, New York.
\par Thesis: \textit{Integrity Models}}
%
\entry*[May 2005] {%
\textbf{B.Sc. Computer Science}, United States Naval Academy, Annapolis, Maryland.
\par Graduated {\it With Merit}, Upsilon Pi Epsilon}

\text{\textbf{Dissertation}
\par Advisor: Aravind Prakash
\par Integrity models are a principled defense mechanism that express a property of well-functioning software, then enforce that property continually at runtime.
Unfortunately, these integrity models are often implemented in a way that compartmentalizes hardware from software.
We present work that unifies these in a single cohesive view.
First, we show that existing hardware resources can be repurposed to support software-layer defenses without onerous impacts to performance.
Then, we present a modified LLVM compiler toolchain used to enforce a diverse body of integrity models through code and data tagging.
Next, we investigate how to optimize a label-based integrity model to minimize binary size increases while maximizing expressiveness of the integrity model.
Finally, we propose using tagging to enforce a read- and write-limited data model, thus bringing forward compile-time data type qualifiers as run-time assurances.}

\end{rubric}