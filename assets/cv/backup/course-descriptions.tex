 \textbf{Course Descriptions}
 \begin{description}
 \item[CS 220] {The architecture and programming of digital computers. Data representation. Processor, memory and I/O organization. Instruction set architectures, encoding and addressing modes. I/O techniques. Interrupts. Assemblers, macro-processors, compilers, interpreters, linkers, loaders. Assembly and machine language programming. C programming language constructs (control and data structures, pointers, arrays and functions) and their relationship to the underlying architecture. Supervised laboratory work involves programming and debugging using machine language, assembly language and C.}
 \item[CS 458, CS 558] {The course provides an introduction to the principles and practices of network, computer, and information security. Topics include authentication and cryptographic techniques, intrusion detection, access control, security policies, and program/policy analysis techniques.}
 \item[CS 480, CS 580] {This hands-on course covers offensive and defensive technologies in the area of software security. Particularly, students will learn about various vulnerabilities that lead to software compromise, attacks that exploit such vulnerabilities, and defenses that defend against such attacks. Topics covered include simple control-flow corruption attacks, slightly harder buffer overflow and return-to-libc attacks, and advanced ROP attacks. Students are expected to not only learn the concepts behind each attack, but also execute them in a controlled environment.}
 \item[CS 580U] {Review of programming concepts, programming environments, debugging tools, large program management and design, program complexity analysis and optimization, and data structures.}
 \end{description}