%!TEX encoding = UTF8
%!TEX root = cv.tex

\begin{rubric}{Experience}

\subrubric{Research Experience}
\entry*[2018\--2022] {\textbf{Research Assistant} Research Foundation for the State University of New York. Binghamton, New York %
\par \begin{list1}
		\item Repurposed Intel Memory Protection Extensions for generalized storage, and implemented analyses for the LLVM compiler toolchain to replace memory accesses with these inlined register operations.
		\item Engineered an implementation of the RISC-V architecture employing inline code for integrity models from compiler-driven static program analysis techniques, using the LLVM compiler toolchain.
		\item Extended our implementation of the RISC-V architecture and LLVM compiler toolchain to perform out-of-band data tagging, with the ability to perform relaxation and linkage resolution at link-time using a modified version of the LLD linker.
		\item Developed a prototype extending data tagging for a read- and write-limited data model that specifically enforces compile-time \texttt{const} qualifiers as run-time assurances.
	\end{list1}
}
\entry*[2015\--2016] {\textbf{Graduate Assistant} Binghamton University. Binghamton, New York%
\par \begin{list1}
	\item Developed a technique for extracting design patterns from C++ source code and encoding as a finite state machine with an XML machine-readable representation
	\item Researched scientific utilization of performance benchmark tools for the computer security domain.
\end{list1}
}
\text{\textbf{Research Interests}
  \par Hardware-software cohesive design for resilient system security through compiler design, program analysis, and reverse engineering.
}

\subrubric{Teaching Experience}
\entry*[2022\--Present] {\textbf{Lecturer} Binghamton University. Binghamton, New York%
\par \begin{list2}%
	\item {Fall 2023 \--- CS 580U: Programming Systems and Tools}
	\item {Fall 2022 \--- CS 580U: Programming Systems and Tools}
	\item {Spring 2022 \--- CS 458, CS 558: Introduction to Computer Security}
\end{list2}
}
\entry*[2017\--2018] {\textbf{Teaching Assistant} Binghamton University. Binghamton, New York%
\par \begin{list2}%
	\item {Spring 2023 \--- CS 458, CS 558: Introduction to Computer Security}
	\item {Spring 2018 \--- CS 480, CS 580: Special Topics: Software Security}
	\item {Fall 2017 \--- CS 220: Computer Systems II, Architecture and C Programming}
\end{list2}
}

\text{\textbf{Courses Prepared to Teach}
  \par \begin{list2}%
  \item Systems programming in C, Rust, and Assembly Languages.
  \item Compilers: lexers, parsers, optimizers, and code generation.
  \item Computer security: authentication and cryptographic techniques, intrusion detection, access control, security policies, reverse engineering, offensive and defensive technologies in software security.
  \item Programming systems and tools with C, C++ or Python.
  \end{list2}
}

% \textbf{Course Descriptions}
% \begin{description}
% \item[CS 220] {The architecture and programming of digital computers. Data representation. Processor, memory and I/O organization. Instruction set architectures, encoding and addressing modes. I/O techniques. Interrupts. Assemblers, macro-processors, compilers, interpreters, linkers, loaders. Assembly and machine language programming. C programming language constructs (control and data structures, pointers, arrays and functions) and their relationship to the underlying architecture. Supervised laboratory work involves programming and debugging using machine language, assembly language and C.}
% \item[CS 458, CS 558] {The course provides an introduction to the principles and practices of network, computer, and information security. Topics include authentication and cryptographic techniques, intrusion detection, access control, security policies, and program/policy analysis techniques.}
% \item[CS 480, CS 580] {This hands-on course covers offensive and defensive technologies in the area of software security. Particularly, students will learn about various vulnerabilities that lead to software compromise, attacks that exploit such vulnerabilities, and defenses that defend against such attacks. Topics covered include simple control-flow corruption attacks, slightly harder buffer overflow and return-to-libc attacks, and advanced ROP attacks. Students are expected to not only learn the concepts behind each attack, but also execute them in a controlled environment.}
% \item[CS 580U] {Review of programming concepts, programming environments, debugging tools, large program management and design, program complexity analysis and optimization, and data structures.}
% \end{description}

\subrubric{Professional Experience}
\entry*[2011\--2014] {\textbf{Instructor}, Trident Training Facility, Bangor, WA.
  \par \begin{list1}
	\item Qualified as Instructor, Instructor Evaluator and Course Supervisor. Served as Navigation Department Director, managing a department of 40 instructors and 11 labs. Awarded Navy and Marine Corps Commendation Medal with Gold Star.
	\item Improved annual throughput in a ship piloting simulator by 18\% (75 sessions) by repairing over 30 script files and qualifying two new instructors. Created an additional 32 trainer sessions per month by guiding a comprehensive lab redesign. 
	\item Delivered lectures for 120 submarine officers annually (66\% increase) and practical skills training for 23 ships, earning a ``highly effective'' rating by external auditors. 
%	\item Cut exam grading and test item analysis time in half (8 hours/graduating class) by building an automated evaluation tool.
\end{list1}
}
\entry*[2011\--2014] {\textbf{Intern}, United States Naval Research Laboratory, Washington, DC.
  \par \begin{list1}
    \item Prototyped a Java Management Extension (JMX) for Mobile Ad Hoc Wireless Networks (MANETs) serving city-sized distributed sensor networks in real-time.
  \end{list1}
}

\end{rubric}
