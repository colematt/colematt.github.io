% Adapted from \url{https://www.stat.berkeley.edu/~paciorek/computingTips/Latex_template_creating_CV_.html}

\documentclass[margin,line]{res}
\usepackage{reslists}
%\usepackage[round]{natbib}	% bibliography package
\usepackage{bibentry}       % full citation in the body of text
\nobibliography* 			% no bib at the end
\usepackage{hyperref}

\oddsidemargin -.5in
\evensidemargin -.5in
\textwidth=6.0in
\itemsep=0in
\parsep=0in

\begin{document}
\name{Matthew R. Cole \vspace*{.1in}}

\begin{resume}

\section{\sc Contact Information}
\vspace{.05in}
\begin{tabular}{@{}p{3in}p{4in}}
P.O. Box 6000                               & {\it Voice:}  (206) 790-8791\\
Department of Computer Science              & {\it E-mail:} \href{mailto:mcole8@cs.binghamton.edu}{mcole8@cs.binghamton.edu}\\
State University of New York at Binghamton  & {\it Web:} \url{https://colematt.github.io}\\
Binghamton, NY 13902-6000                   &
\end{tabular}

\section{\sc Research Interests}
Computer system security at the intersection of compiler design, program analysis, computer architecture, and reverse engineering.

\section{\sc Education}
{\bf State University of New York at Binghamton}, Binghamton, New York USA\\
%{\em Department of Computer Science}
\vspace*{-.1in}
\begin{list1}
\item[] Ph.D. Candidate, Computer Science
\begin{list2}
\vspace*{.05in}
\item Dissertation Topic: ``Enforcing Integrity Models Through Hardware-Software Cohesive Systems''
\item Advisor: Aravind Prakash
\end{list2}
\vspace*{.05in}
\item[] M.S., Computer Science,  May 2018
\begin{list2}
\vspace*{.05in}
\item Thesis Topic: ``Integrity Models''
\end{list2}
\end{list1}

{\bf United States Naval Academy}, Annapolis, Maryland USA\\
%{\em Department of Computer Science}
\vspace*{-.1in}
\begin{list2}
\item[] B.S. Computer Science,  May, 2005
\end{list2}

\section{\sc Honors and Awards}

United States Naval Academy: graduated \textit{With Merit}, Upsilon Pi Epsilon. Captained winning team of National Security Agency's Cyber Defense Exercise.

%United States Navy: Defense Meritorious Service Medal, Navy \& Marine Corps Commendation with Gold Star, Navy \& Marine Corps Achievement Medal, Afghanistan Campaign Medal.

\section{\sc Academic Experience}

{\bf State University of New York at Binghamton}, Binghamton, New York USA

\vspace{-0.3cm}
{\em Lecturer} \hfill {\bf January 2022 \-- Present} \\
Taught graduate and undergraduate courses for classes between 30 and 70 students.

\begin{list2}
\item CS 580U: Programming Systems and Tools, Fall 2022
\item CS 458, CS 558: Introduction to Computer Security, Spring 2022
\end{list2}

\vspace{-0.1cm}
{\em Research Assistant} \hfill {\bf January, 2018 - December 2021}\\
Performed analyses using the LLVM compiler toolchain, providing defenses to eliminate attack surface area. Engineered an implementation of the RISC-V architecture employing inline code and data tagging for integrity models.

\vspace{-.1cm}
{\em Teaching Assistant} \hfill {\bf January, 2017 \-- Present}\\
Co-taught graduate and undergraduate level courses. Authored and proctored weekly labs and graded all coursework. Delivered lectures during instructor-of-record's absence. Piloted a Github Classroom/Travis-CI course delivery system to provide instant feedback and version control software experience to students while expediting grading.

\vspace*{.05in}
\begin{list2}
\item CS 458, CS 558: Introduction to Computer Security, Spring 2023
\item CS 480, CS 580: Introduction to Computer Security, Spring 2017
\item CS 220: Computer Systems II, Architecture and C Programming, Fall 2017
\end{list2}

\vspace{-.1cm}
{\em Graduate Assistant} \hfill {\bf August, 2015 - January 2017}\\
Explored widespread unscientific use of performance benchmarks within the computer security community. Repurposed Intel's MPX spatial memory safety architecture extension as secure storage for information hiding applications.

\section{\sc Publications}

%\bibentry{quach2017supplementing} \\
%\bibentry{cole2022simplex} \\
%\bibentry{gollapudi2023control} \\

% \section{\sc Conference Presentations}

%Paciorek, C.J. and R. Rosenfeld.  
%Minimum classification error training in exponential language models.  
%2000 Spring Transcription Workshop, College Park, Maryland.
% \vspace*{-.25in}
% \begin{verbatim}http://www.nist.gov/speech/publications/tw00/html/abstract.htm#cp1-50\end{verbatim}

\section{\sc Professional Experience}

{\bf United States Navy}, Bangor, Washington USA

\vspace{-.3cm}
{\em Department Director} \hfill {\bf May, 2005 - July, 2014}\\
Held qualifications as 
%Submarine Officer, Nuclear Engineering Officer, Joint Planning Officer, 
Instructor, Instructor Evaluator and Course Supervisor. Oversaw a department of 40 instructors and 11 laboratories. Planned curricula for 3 courses and delivered lectures for over 120 trainees and 23 submarine crews annually. Rated ``highly effective'' by external auditors during entire tenure.

{\bf United States Naval Research Laboratory}, Washington, District of Columbia USA

\vspace{-.3cm}
{\em Intern} \hfill {\bf May, 2004 - August, 2004}\\
Prototyped Java Management Extensions (JMX) for Mobile Ad Hoc Wireless Networks (MANETs) serving large, distributed sensor networks in real-time.

%\section{\sc Computing Skills}
%\begin{list2}
%\item Languages: C, C++, Python, Assembly (x86 and RISC-V), LLVM IR.
%\item Operating Systems: Unix/Linux, MacOS.
%\item Benchmark Frameworks: SPEC CPU 2006/2017, Hayai, Google Benchmark.
%\item Test Frameworks: CUnit, Google Test, Boost Test, Python unittest.
%\item Applications: LLVM Project, Git, Travis-CI, CMake, GNU Build System, \LaTeX.
%\end{list2}

\section{\sc Professional Service}
\begin{list2}
\item Binghamton University Graduate Student Organization Senate, 2016-2018.
\item Binghamton University Graduate Student Organization Judicial Officer, 2018-2019.
\item ACSAC Artifact Committee, 2017. \url{https://www.acsac.org/2017/committees/#artifact}
\item ACSAC Artifact Committee, 2020. \url{https://www.acsac.org/2020/committees/artifact/}
\item ACSAC Artifact Committee, 2021. \url{https://www.acsac.org/2021/committees/artifact/}
\end{list2}

\end{resume}

\end{document}




