\documentclass[letterpaper,nolmodern]{moderncv}
  \moderncvtheme[blue]{classic}
\usepackage[utf8]{inputenc}
\usepackage[scale=0.9]{geometry}
  \recomputelengths


\firstname{Matthew}
\lastname{Cole}
\title{Curriculum Vitae}
\address{Department of Computer Science\\%
Thomas J. Watson College of Engineering and Applied Sciences\\%
4400 Vestal Parkway East\\}%
{Binghamton, NY 13902-6000}
\mobile{\href{tel:12067908791}{+1 (206) 790-8791}}
\email{mcole8@binghamton.edu}
\extrainfo{Web: \url{https://colematt.github.io}\\% 
\url{https://github.com/colematt}\\}

\begin{document}

\maketitle

\section{Education}
\cventry{2018\--present}{Ph.D. Computer Science}{State University of New York at Binghamton}{Binghamton, New York USA}{expected May 2024}{}
\cventry{2015\--2018}{M.S. Computer Science}{State University of New York at Binghamton}{Binghamton, New York USA}{}{GPA: 3.97/4.00}
\cventry{2001\--2005}{B.S. Computer Science}{United States Naval Academy}{Annapolis, Maryland USA}{\textit{With Honors}}{GPA: 3.78/4.00 (major), 3.60/4.00 (overall)}

\subsection{Doctoral Thesis}
\cvline{Title}{\emph{Integrity Models}}
\cvline{Supervisor}{Aravind Prakash}
\cvline{Committee}{Kanad Ghose, Kartik Gopalan, Dmitry Ponomarev}
\cvline{Description}{\small Short thesis abstract}

\subsection{Master Thesis}
\cvline{Title}{\emph{Integrity Models}}
\cvline{Supervisors}{Aravind Prakash}
\cvline{Description}{\small Short thesis abstract}

%\section{Publications}
\nocite{*}
\bibliographystyle{plain}
\bibliography{publications}
\emptysection{}\closesection

\section{Experience}
\subsection{Research Experience}
\cventry{January 2018\--Present}{Research Assistant}{Research Foundation for the State University of New York}
     {Binghamton, New York}{}
     {
     	\cvlistitem{point1}
     	\cvlistitem{point1}
     	\cvlistitem{point1}
     }
\subsection{Teaching Experience}
\cventry{January 2018\--Present}{Lecturer}{State University of New York at Binghamton}
     {Binghamton, New York}{}
     {
     	\cvlistitem{Fall 2023 \--- CS 580U: Programming Systems and Tools}
     	\cvlistitem{Fall 2022 \--- CS 580U: Programming Systems and Tools}
     	\cvlistitem{Spring 2022 \--- CS 458, CS 558: Introduction to Computer Security}
     }
\cventry{January 2018\--Present}{Teaching Assistant}{State University of New York at Binghamton}
     {Binghamton, New York}{}
     {
     	\cvlistitem{Spring 2023 \--- CS 458, CS 558: Introduction to Computer Security}
     	\cvlistitem{Spring 2017 \--- CS 480, CS 580: Special Topics: Software Security }
     	\cvlistitem{Fall 2017 \--- CS 220: Computer Systems II, Architecture and C Programming}
     }

\textbf{Course Descriptions}
\begin{description}
	\item[CS 220] The architecture and programming of digital computers. Data representation. Processor, memory and I/O organization. Instruction set architectures, encoding and addressing modes. I/O techniques. Interrupts. Assemblers, macro-processors, compilers, interpreters, linkers, loaders. Assembly and machine language programming. C programming language constructs (control and data structures, pointers, arrays and functions) and their relationship to the underlying architecture. Supervised laboratory work involves programming and debugging using machine language, assembly language and C.
	\item[CS 458, CS 558] The course provides an introduction to the principles and practices of network, computer, and information security. Topics include authentication and cryptographic techniques, intrusion detection, access control, security policies, and program/policy analysis techniques.
	\item[CS 480, CS 580] This hands-on course covers offensive and defensive technologies in the area of software security. Particularly, students will learn about various vulnerabilities that lead to software compromise, attacks that exploit such vulnerabilities, and defenses that defend against such attacks. Topics covered include simple control-flow corruption attacks, slightly harder buffer overflow and return-to-libc attacks, and advanced ROP attacks. Students are expected to not only learn the concepts behind each attack, but also execute them in a controlled environment.
	\item[CS 580U] Review of programming concepts, programming environments, debugging tools, large program management and design, program complexity analysis and optimization, and data structures.
\end{description}

\subsection{Professional Experience}

\section{Research Interests}

Hardware-software cohesive design for resilient system security through compiler design, program analysis, and reverse engineering.

\section{Skills}
\subsection{Computing}
\cvcomputer{Languages}{C, C++, Python, x86/RISC-V/ARM assembly, LLVM IR, Lisp, Rust, Java}%
{Operating Systems}{Unix/Linux (Ubuntu, Debian, Kali), MacOS, Windows}%
\cvcomputer{Benchmarking}{SPEC CPU 2006/2017, Hayai, Google Benchmark}%
{Testing}{CUnit, Google Test, Boost.Test, Python unittest, LLVM Lit}%
\cvcomputer{Build}{GNU Make, CMake, Github Actions, TravisCI}%
{Reverse Engineering}{Ghidra, Hopper}%
\cvline{Other}{Doxygen, Git}

%\subsection{Languages}
%\cvlanguage{English}{Fluent (native)}{}
%\cvlanguage{French}{Basic}{}

\section{Service}
\subsection{Professional Service}
\subsection{University Service}

\emptysection{}\closesection
\vfill
\begin{center}
Last Updated: \today{}
\end{center}

\end{document}
